\section{Preliminaries and Mathematical Notations} \label{sec:prelims}
In this work, we represent scalars in plain, vectors by lower-case bold letters (e.g., $\mathbf{a}$) and matrices by upper-case bold letters (e.g., $\mathbf{A}$). Ring elements, for example  $x \in R_q$, are represented by lower case letters in plain. We define $\left[k\right] = \left\{1, \cdots , k\right\}$ for any non-negative integer $k$. The $i$-th norm of a vector $\vec{\mathbf{v}}$ is denoted by $\norm{\mathbf{v}}_i$. We denote the tensor (Kronecker) product of two matrices $\mathbf{A}$ and $\mathbf{B}$ as $\mathbf{A} \otimes \mathbf{B}$. We extend this notation to a vector $\vec{\mathbf{v}}$ where the Kronecker product is represented as  $\vec{\mathbf{v}} \otimes \mathbf{A}$. Unless explicitly mentioned logarithms are to be understood with base 2. We denote the horizontal and vertical concatenation of matrices by operators $\left(\cdot||\right)$  and $\left( \cdot ||^\intercal \right)$ respectively. For two matrices $\mathbf{A},\mathbf{B} \in \mathbb{Z}^{n \times n}$ $\left[ \mathbf{A} \; || \; \mathbf{B} \right]$ produces a matrix $\mathbf{C} \in \mathbb{Z}^{n \times 2n}$. Similarly, $\left[ \mathbf{A} \; ||^\intercal \; \mathbf{B} \right]$ produces a matrix $\mathbf{C} \in \mathbb{Z}^{2n \times n}$.


\subsection{Gadget Matrix and Relinearization functions:}

For LWE dimension $n$ and modulus $q$, we use the following ``gadget'' vector \cite{micciancio2012trapdoors}: 
$$\mathbf{g} = \left(1,2,4,\cdots,2^{\ell-1} \right) \in \mathbb{Z}_q^{\ell}, \text{ where } \ell = \lceil \log q \rceil.$$
The \emph{gadget matrix} \textbf{G} is then defined as the diagonal concatenation of the \textbf{g} vector $n$ times. Formally, it is written as $ \mathbf{G} = \mathbf{g} \otimes I_n \in \mathbb{Z}_q^{\ell n\times n}$. To perform relinearization we define the following operations \cite{brakerski2011fully, gentry2013homomorphic} with respect to an element $a$ which can be either a vector or matrix or polynomial ring. 

\begin{itemize}
\item{$\mathbf{BitDecomp}\left(a\right)$}: For a vector $\vec{\mathbf{a}} \in \mathbb{Z}_q^n$ this operation produces a bit decomposed and expanded vector $\vec{\mathbf{a}}' \in \mathbb{Z}_q^{n\ell}$ where $a_i = \sum^{\ell}_{j=0}2^j a'_{i\ell + j}$. In case of a matrix $\mathbf{A} \in \mathbb{Z}_q^{m\times n}$ this operation produces a matrix that is expanded along the column resulting in $\mathbf{A}' \in \mathbb{Z}_q^{m\times n\ell}$. Finally, in case of a polynomial ring $a \in R_q$ this operation produces $\mathbf{a}' \in R_q^\ell$.  
\item{$\mathbf{PowerOf2}\left(a\right)$}: Given a vector $\vec{\mathbf{a}} \in \mathbb{Z}_q^n$ this operation produces an expanded vector $\vec{\mathbf{a}}' \in \mathbb{Z}_q^{n\ell}$ where $\vec{\mathbf{a}}' = \left(a_0,2a_0,\cdots,2^{\ell-1}a_0,\cdots, 2^{\ell-1}a_{n-1}\right)$. Similarly, for a polynomial ring $a \in R_q$ we get $\vec{\mathbf{a}}' \in R_q^{\ell}$ where $a'_i = 2^ia$.
\end{itemize}
Using these operations, we can produce a product of $\vec{\mathbf{a}}$ and $\vec{\mathbf{b}}$ as follows: $$\left\langle \mathbf{BitDecomp}\left( \vec{\mathbf{a}} \right), \mathbf{PowersOf2}\left( \vec{\mathbf{b}}\right)\right\rangle = \left\langle \vec{\mathbf{a}},\vec{\mathbf{b}}\right\rangle$$ 
